\documentclass[12pt]{article}
\usepackage[total={18cm,22cm},top=3cm,left=2cm,right=2cm]{geometry}
\parindent=0mm
\usepackage[utf8]{inputenc}

\usepackage{booktabs}

\usepackage[table]{xcolor}

\usepackage{multicol}

\usepackage{graphicx}

\usepackage{amsmath}

\usepackage{tikz}
\usepackage{enumitem}
\definecolor{azulF}{rgb}{.0,.0,.3}
\newcommand{\cnumero}[2]{\tikz[baseline=(myanchor.base)]
\node[minimum size=0.2cm,circle,
inner sep=1pt,draw, #2, thick, fill=#2](myanchor)
{\color{white}\bfseries\fontsize{8}{8}#1};}

\newcommand*{\itembolasazules}[1]{\protect\cnumero{#1}{azulF}}

\usepackage{fancyhdr}
\pagestyle{fancy}
\fancyhf{}
%TODO
\fancyhead[L]{Teoría de Sistemas - Grupo Fénix}
%TODO
\fancyhead[R]{UNAH-VS}
\fancyfoot[C]{\thepage}

\renewcommand*\contentsname{Contenido}

\begin{document}
%*******************************************************
% Portada
%*******************************************************
\begin{titlepage}

  \begin{center}
    {\includegraphics[width=0.65\linewidth]{$HOME/Dev/dotfiles/VimTempls/Imgs/unahvs-logo.png}\par}
  
    {\bfseries\Huge Universidad Nacional Autónoma de\\
                     Honduras en el valle de Sula \par}
  
    \vspace{1cm}
  
%TODO
    {\scshape\huge TEORÍA DE SISTEMAS\par}
  
    \vspace{1cm}
  
%TODO
    {\scshape\Large Identificación de Problemas,\\
              Síntomas, Soluciones Propuestas }
  
    \vfill 
    {\large Catedrático:\par} %TODO
    {\large Lic. Tania Pineda\par} %TODO
  
    \vfill

      \textbf{\textit{ Grupo Fénix }}
    
      \begin{tabular}{lr} \toprule
          Nombre Completo & No. Cuenta\\ \midrule
          Areli Yaniris Flores Soto & 20172030394 \\
          Paola Maria Pérez Reyes & 20152030878 \\
          Samuel Alejandro Suazo Méndez & 20182000862 \\
          Jimmy Xavier Triminio Hernández & 20122008197 \\
          Luis David Romero Enamorado & 20182001663 \\ \bottomrule
      \end{tabular} 

    \vfill
    {\large OCTUBRE 2021 | UNIDAD \#2 \par} %TODO

  \end{center}

\end{titlepage}

%*******************************************************
% Contenido 
%*******************************************************
\newpage
\tableofcontents

%*******************************************************
% Introducción
%*******************************************************
\newpage
\section{ Introducción }
\vspace{1cm}

\begin{center}
\begin{minipage}{12cm}

  El analista tiene a ser es observador que solo pervive los diferentes 
  problemas a las que se enfrentas las empresas y como estas tiene que ver
  como dar solución a estos. y que son claros interruptores en off que 
  hacer muchas de las área que depende la información clara, transparente
  y a tiempo, para que estos puede tomar acción en tiempo y forma.

\end{minipage}
\end{center}

%*******************************************************
% Objetivos 
%*******************************************************
\newpage
\section{ Objetivos }
\vspace{1cm}

\begin{itemize}
  \item Realizar el análisis de como una empresas puede llegar a agilizar
    las diferentes formas de gestionar su información.
  \item Veremos que la utilización de las bases de datos son muy importante.
  \item Ver los puntos mas débiles que todas las empresas llegar ha tener 
    entre sus filas. Como es la comunicación entre las diferentes áreas.
\end{itemize}

% Tema del documento 
%*******************************************************
\newpage

\begin{center}
\section{ 
Identificación de Problemas, Síntomas,
Soluciones Propuestas
}

\vspace{1cm}

\end{center}

\subsubsection{ Planteamiento del caso de Estudio }

\vspace{.25cm}
\begin{center}
\begin{minipage}{14cm}

  Con miras de realizar el desarrollo analítico de las 
  posibles \textit{ resoluciones } al caso de estudio de Concrex,
  (\textit{Concretos Éxito}), en que la empresas como tal carece de un 
  sistemas de control para gestionar la exigencias de la materias primas 
  para la variedad de productos del cual cuenta en su catalogo de productos 
  y servicios, y de esta manera optimizar sus procesos internos, 
  el \textit{ análisis } de las compras realizadas, y todas las actividades 
  de desarrollan al rededor de las ventas de bienes y servicios.\\

  En el \textit{ análisis } nos muestra la \textit{problemática} en un
  contexto de distribución, inventario y \textit{ comunicación } entre los
  diferentes departamentos que esta involucrados al momento de realizar 
  las actividades, de donde podemos notar el poco desarrollo de 
  procesos sistematizados que pueden minimizar dichos procesos como ser:

  \begin{itemize}
    \item La \textit{ comunicación }.\\
      En la \textit{comunicación} entre la oficina principal y sus empleados,
      la oficina principal y los \textit{Jefes Área} y los despachadores
      finales, los que afecta directamente a la movilización de los pedidos.
    \item Sistema de gestión de inventarios. \\
      No se cuenta con un sistema de gestión o manejo de inventarios stock,
      (misceláneos, materias primas, producto final) y que este permita a
      demás una mejor distribución entre los items de productos que tengan 
      mayor y menor demanda, y saber los minios y máximos de los mismos.

  \end{itemize}
  

\end{minipage}
\end{center}

\vspace{0.5cm}

\subsubsection{ Sintomas del Problema }
\begin{center}
\begin{minipage}{14cm}

  En primer lugar, nos centraremos el ver el largo procesos que se lleva a
  cabo y de como fluye la información entre los diferentes departamentos.
  Ya que al momento de realizar la tramitación ya sea de una compra de 
  materia prima o servicios, puede llegar varias horas, para llegar ser
  procesado y efectuado, entre 2 a 3 días, en tanto cotizaciones, firmas 
  autorizadas, realizar la compra, tiende a llegar mucho tiempo y si hablamos
  de salas de ventas o proyectos que están algo retirados, el tiempo que
  se puede tomar llegaríamos a hablar de semanas.\\

  Por otra parte, vemos la dificulta de llegar el seguimientos del inventario
  y de todos lo que aplica para ello, y recordando que cuando hablamos de 
  productos que puede llegar a presentar deterioro, es importante tener en
  cuentas que \textit{"primeras entradas son las primeras salidas"}, y tener
  al día las fechas de caducidad de aquellos productos que puedan presentar
  esta característica perecedera, y entres otros el proceso sea más incomodo,
  tedioso y extenso.

\end{minipage}
\end{center}

\vspace{0.5cm}

\subsubsection{ Identificación de Problemas}
\begin{center}
\begin{minipage}{14cm}
  \begin{itemize}
    \item No se cuenta con un sistema de gestión para inventarios.
    \item No se planifica correctamente la distribución de las compras.
    \item No existen una buena comunicación entre los empleados.
    \item No hay comunicación entre las diferentes áreas del ciclo de ventas
      y producción. 
  \end{itemize}

\end{minipage}
\end{center}

\vspace{0.5cm}

\subsubsection{ Soluciones Propuestas}
\begin{center}
\begin{minipage}{14cm}
  \begin{itemize}
    \item Crear un sistema de información donde se pueda controlar el
      inventario, las compras y los diferentes pedidos.
    \item Establecer alineamientos o políticas que sean flexibles
      pero que a su vez no den libre modificación una vez realizados los
      pedidos y una gestión de reclamos o devoluciones.
    \item Establecer un periodo de registro histórico
      \textit{(base de datos)} que permita el seguimiento del analítico de
      las ventas y estados financieros de clientes, proveedor y acreedores.
  \end{itemize}

\end{minipage}
\end{center}
\vspace{2cm}


\end{document}
end{document}

