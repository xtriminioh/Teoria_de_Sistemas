\documentclass[12pt]{article}
\usepackage[total={18cm,22cm},top=3cm,left=2cm,right=2cm]{geometry}
\parindent=0mm
\usepackage[utf8]{inputenc}

\usepackage{multicol}

\usepackage{graphicx}

\usepackage{amsmath}

\usepackage{hyperref}

\usepackage{tikz}
\usepackage{enumitem}
\definecolor{azulF}{rgb}{.0,.0,.3}
\newcommand{\cnumero}[2]{\tikz[baseline=(myanchor.base)]
\node[minimum size=0.2cm,circle,
inner sep=1pt,draw, #2, thick, fill=#2](myanchor)
{\color{white}\bfseries\fontsize{8}{8}#1};}

\newcommand*{\itembolasazules}[1]{\protect\cnumero{#1}{azulF}}

\usepackage{fancyhdr}
\pagestyle{fancy}
\fancyhf{}
%TODO
\fancyhead[L]{TEORÍA DE SISTEMAS}
%TODO
\fancyhead[R]{UNAH-VS}
\fancyfoot[C]{\thepage}

\renewcommand*\contentsname{Contenido}

\begin{document}
%*******************************************************
% Portada
%*******************************************************
\begin{titlepage}

  \begin{center}
    {\includegraphics[width=0.65\linewidth]{/home/xtriminio/.config/nvim/template/Imgs/unahWallpaper.png}\par}

    
  
    {\bfseries\Huge Universidad Nacional Autónoma de\\
                     Honduras en el valle de Sula \par}
  
    \vspace{1cm}
  
%TODO
    {\scshape\huge TEORÍA DE SISTEMAS\par}
  
    \vspace{1cm}
  
%TODO
    {\scshape\Large COMPUTACIÓN EN LA NUBE.}
  
    \vfill 
    {\large Catedrático:\par} %TODO
    {\large Ing. Tania Melissa Pineda Godoy \par} %TODO
  
    \vfill
    {\large Alumno: \par}
    {\large Jimmy Xavier Triminio Hernández - 20122008197 \par}

    \vfill
    {\large Febrero 2023\par} %TODO

  \end{center}

\end{titlepage}

%*******************************************************
% Contenido 
%*******************************************************
\newpage
\tableofcontents

%*******************************************************
% Introducción
%*******************************************************
%\newpage
\vspace{1cm}
\section{ Introducción }

  La computación en la nube es un concepto que incorpora
  el software como servicio, como en la web 2.0 y otros
  conceptos recientes, también conocidos como servicios
  en la nube. Estos significa que los usuarios pueden
  utilizar recursos como computación, almacenamiento y 
  aplicaciones a través de internet, sin la necesidad 
  de instalar y administrar software en sus propios 
  sistemas. La computación en la nube ofrece servicios
  como almacenamiento de datos, seguridad, redes,
  aplicaciones de software, y business intelligence a
  través de la internet. También puede permitir a los 
  usuarios el acceso a recursos informáticos a nivel
  mundial sin la necesidad de invertir en un sistema
  local. Esto permite a los usuarios crear, implementar
  y ejecutar aplicaciones de forma rápida, segura y 
  escalable.

%*******************************************************
% Contenido
%*******************************************************
\vspace{1cm}
\section{ La Computación en la Nube}
  Los inicios de la \textit{computación en la nube} se 
  remontan a finales de los años 90, cuando se empezó a
  hablar sobre el concepto de una "red de computadoras
  universales". El concepto básico del \textbf{cloud computing}
  o computación en la nube fue acuñado en un seminario
  impartido por Ramalingam Chellappa \footnote{\textbf{Ramalingam Chellappa}: es un profesor distinguido den Bloomberg, que trabaja en la Universidad Johns Hopkis. Es el científico en jefe del Instituto Johns Hopkis para la Autonomía Asegurada y se unió a la misma después de 29 años en la Universidad de Maryland.} 
  en 1997 se le atribuye
  el término a múltiples autores, entre ellos Carl Robnett Licklider \footnote{\textbf{Carl Robnett Licklider}: conocido simplemente como JCR o "Lick", fue un informático estadounidense que se considera estar entre las figuras más prominentes en el desarrollo de las ciencias de la computación y la historia general de la computación}, 
  quien trabajó para el Departamento
  de Defensa de los Estados Unidos en los años 60. En ese
  momento, la idea era crear una red de computadoras que
  pudieran compartir recursos entre sí, permitiendo a los
  usuarios acceder a los mismos recursos desde grandes
  plataformas de almacenamiento en la nube, como 
  \href{https://aws.amazon.com/es/}{Amazon Web Services} y
  \href{https://azure.microsoft.com/en-us/pricing/purchase-options/pay-as-you-go/}{Microsoft Azure},
  y la computación en la nube se ha convertido en uno de los 
  principales motores de economía digital.

%*******************************************************
% Biografia
%*******************************************************
\newpage
\section{ Biografía }
\vspace{1cm}

\begin{enumerate}
  \item \textbf{Introducción a la computación en la nube}\\
    \href{https://courses.minnalearn.com/es/courses/digital-revolution/the-cloud-computing-revolution/introduction-to-cloud-computing/}{courses.minnalearn.com}

  \item \textbf{Servicio de Amazon AWS}\\
    \href{https://aws.amazon.com/es/}{aws.amazon.com}

  \item \textbf{Microsoft Azure}\\
    \href{https://azure.microsoft.com/en-us/pricing/purchase-options/pay-as-you-go/}{azure.microsoft.com}

  \item \textbf{Google Sholar} - \textit{Ramalingam Chellappa}\\
    \href{https://scholar.google.com/citations?user=L60tuywAAAAJ&hl=en}{sholar.google.com}

  \item \textbf{JCR}\\
    \href{https://en.wikipedia.org/wiki/J._C._R._Licklider}{en.wikipedia.org}

\end{enumerate}


\end{document}

