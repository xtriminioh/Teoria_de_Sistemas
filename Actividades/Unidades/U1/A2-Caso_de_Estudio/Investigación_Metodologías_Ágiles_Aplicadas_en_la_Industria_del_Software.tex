\documentclass[12pt]{article}
\usepackage[total={18cm,22cm},top=3cm,left=2cm,right=2cm]{geometry}
\parindent=0mm
\usepackage[utf8]{inputenc}

\usepackage{hyperref}

\usepackage{multicol}

\usepackage{graphicx}

\usepackage{amsmath}

\usepackage{tikz}
\usepackage{enumitem}
\definecolor{azulF}{rgb}{.0,.0,.3}
\newcommand{\cnumero}[2]{\tikz[baseline=(myanchor.base)]
\node[minimum size=0.2cm,circle,
inner sep=1pt,draw, #2, thick, fill=#2](myanchor)
{\color{white}\bfseries\fontsize{8}{8}#1};}

\newcommand*{\itembolasazules}[1]{\protect\cnumero{#1}{azulF}}

\usepackage{fancyhdr}
\pagestyle{fancy}
\fancyhf{}
%TODO
\fancyhead[L]{TEORÍA DE SISTEMAS}
%TODO
\fancyhead[R]{UNAH-VS}
\fancyfoot[C]{\thepage}

\renewcommand*\contentsname{Contenido}

\begin{document}
%*******************************************************
% Portada
%*******************************************************
\begin{titlepage}

  \begin{center}
    {\includegraphics[width=0.65\linewidth]{$HOME/Imágenes/Imgs/unahvs-logo.png}\par}
  
    {\bfseries\Huge Universidad Nacional Autónoma de\\
                     Honduras en el valle de Sula \par}
  
    \vspace{1cm}
  
%TODO
    {\scshape\huge TEORÍA DE SISTEMAS\par}
  
    \vspace{1cm}
  
%TODO
    {\scshape\Large 
      Investigación Metodologías Ágiles Aplicadas en la Industria del Software.
    }
  
    \vfill 
    {\large Catedrático:\par} %TODO
    {\large Ing. Tania Melissa Pineda Godoy\par} %TODO
  
    \vfill
    {\large Alumno: \par}
    {\large Jimmy Xavier Triminio Hernández - 20122008197 \par}

    \vfill
    {\large FEBRERO 2023\par} %TODO

  \end{center}

\end{titlepage}

%*******************************************************
% Contenido 
%*******************************************************
\newpage
\tableofcontents

%*******************************************************
% Introducción
%*******************************************************
\newpage
\section{ Introducción }

\vspace{1cm}
\begin{center}
  \begin{minipage}{14cm}
  La industria del software está cada vez más interesada en el uso de metodologías ágiles
  para desarrollar software de calidad. Estas metodologías ofrecen una serie de
  beneficios, como el incremento de la productividad, la reducción de costos, la mejora de la
  calidad del producto y la rapidez con la que se llevan a cabo los proyectos. Esto ha
  llevado a un aumento en la investigación sobre el uso de metodologías ágiles en la
  industria del software. En este documento, se presenta una breve introducción a la 
  investigación de metodologías ágiles aplicadas a la industria del software. Se discutirán
  los conceptos básicos de la metodología ágil, los beneficios y desventajas de su uso, así
  como los últimos avances en el tema. Se concluirá con una lista de recursos útiles para 
  aquellos interesados en el tema.
  \end{minipage}
\end{center}


\newpage

\section{ Objetivos }

\begin{enumerate}
  \item Conocer las habilidades para trabajar de manera ágil en un entorno empresarial.
  \item Comprender los principios básicos de la metodología ágil.
  \item Conocer los valores de la metodología ágil.
  \item Conocer el proceso de la metodología ágil.
  \item Aprender cómo implementar técnicas y herramientas para aplicar la metodología en el trabajo.
\end{enumerate}

%\newpage
\vspace{1cm}

\section{Investigación Metodologías Ágiles Aplicadas en la Industria del Software.}

\vspace{1cm}
\subsection{ Metodología Ágil }

Metodología Ágil es un conjunto de prácticas y técnicas para la gestión de proyectos que 
ofrecen rapidez, eficiencia y flexibilidad. Su objetivo principal es aumentar la 
calidad del producto final, ofreciendo resultados inmediatos y ahorrando tiempo y recursos. 
Esta metodología se caracteriza por el uso de sprints cortos, donde los equipos trabajan en 
pequeños incrementos de código para entregar resultados rápidamente. Además, se enfoca en 
la colaboración, el auto-aprendizaje y la retroalimentación continua. Esto se logra a través 
del uso de herramientas ágiles como el seguimiento de tareas, reuniones de standup y 
retrospectivas. Esto permite a los equipos entregar productos de mayor calidad de manera 
más rápida.\\

Los métodos ágiles son una colección de metodologías innovadoras para el desarrollo 
de sistemas las cuales se centran en los usuarios los cuales son en principio el 
cliente o usuario final, de cada uno de los productos finales, obtenidos de aplicar estas metodologías para desarrollo de {\it software } especializados.\\ 

\vspace{1cm}
\subsection{ Principios básicos de la metodología ágil }
Para poder explicar este método es imprescindible ceñirse a estos 12 principios fundamentales:

\begin{enumerate}
  \item Perseguir las \textit{satisfacción del cliente e informarle periódicamente} del estado del proyecto
  \item Los nuevos \textit{cambios y requisitos son bienvenidos} y se valoran como modificaciones positivas
  \item La división del trabajo se realiza en fases \textit{fases temporales productivas} divididas en semanas, quincenas, etc
  \item Posibilidad de \textit{medir el progreso}
  \item La forma de ejecutar los proyectos debe \textit{garantizar en sí misma la continuidad del proyecto} (desarrollo sostenible)
  \item \textit{El equipo debe trabajar de forma coordinada y en conjunto}, utilizando el \textit{método Scrum} \footnote{\textbf{Método Scrum} es una técnica de metodología ágil ampliada que ofrece una forma de conectar varios equipos que necesitan trabajar juntos para ofrecer soluciones complejas.}, como una práctica efectiva y esencial para la correcto organización y desarrollo del trabajo.
  \item \textit{Las conversaciones entre los integrantes del equipo y/o cliente deben llevarse a cabo en persona}, para comunicar de forma eficaz los mensajes.
  \item Es necesario \textit{infundir motivación y confianza a los miembros} que forman parte del proyecto para obtener procesos exitosos.
  \item \textit{Excelencia técnica y buen diseño}. En la metodología ágil, la calidad trabajo y la presentación forman parte del conjunto.
  \item Se impone la \textit{ley de la simplicidad}. Las tareas deben ser lo más sencillas posibles. En caso de no poder simplificar se tendrá que dividir en iteraciones para reducir su nivel de complejidad.
  \item \textit{Equipos auto-gestionados}. Aunque es necesario que existan una figura que monitorice los equipos de trabajo, ésto deben ser capaces de organizarse por sí mismos.
  \item \textit{Adaptación a las circunstancias cambiantes}. Es imprescindible que los profesionales que ejecute los proyectos puedan adaptarse a las distintas circunstancias y modificaciones que puedan surgir durante el proceso.

\end{enumerate}

\vspace{1cm}
\subsection{ Valores de la metodología ágil }

El método ágil no solo se basa en el resultado final, sino que también en los
valores, principios y prácticas. Los {\it valores} y {\it principios} establecidos
son esenciales para la programación ágil; estos crean el contexto para la
colaboración entre programadores y clientes. La metodología ágil es una metodología de desarrollo de \textit{software} que se basa en valores, principios y prácticas básicas. Los cuatro valores de la metodología ágil son: {comunicación}, \textit{simpleza}, \textit{retroalimentación} y \textit{valentía}.\\

Como a menudo hay tensión entre los que hacen los desarrolladores a corto plazo
y los que lo comercialmente deseable a largo plazo, es importante establecer los valores que formarán la base para actuar en conjunto en un proyecto de {\it software }. Estos valores son los siguientes:

\begin{enumerate}
  \item {\bf Comunicación }\\
    En todo esfuerzo humano existe la posibilidad de una mala comunicación.
    Los proyectos de sistemas que requieren de una constante actualización de una
    constante actualización y diseño técnico son especialmente propensos a dichos 
    errores. Si sumamos a ello tiempo de entrega ajustados, jerga especializada y 
    el estereotipo de que los programadores prefieren hablar con las maquinas en vez
    de las personas, terminamos con el potencial de toparnos con serios problemas
    de comunicación.

  \item {\bf Simpleza }\\
    Cuando trabajamos en un proyecto de desarrollo de {\it software}, nuestra primera
    tendencia es abrumarnos con la complejidad y tamaño de la tarea.
    La {\it simpleza } para el desarrollo de {\it software} significa que debemos 
    empezar con las cosas más simples que podamos realizar. Y el valor de la simpleza
    nos pide hacer las cosas más sencillas hoy, a sabiendas de que tal vez mañana 
    tengamos que cambiarla un poco.

    ocurrir en cuestión de segundos, minutos, días, semanas o meses, esto
    dependiendo de lo que se requiera, de quién se esté comunicando y de lo que se
    pretende hacer con la retroalimentación.

  \item {\bf Retroalimentación }\\
    La retroalimentación es una forma de control de un sistema. Como sistemas, todas organizaciones usan la planeación y el control de para administrar sus recursos con efectividad.
    La retroalimentación se recibe desde el interior de la organización y de los entornos
    exteriores. Cualquier cosa externa a los límites de una organización se considera un entorno. Numerosos entornos con diversos grados de estabilidad constituyen el medio en el que existen las organizaciones.


  \item {\bf Valentía }\\
    El valor de la valentía tiene que ver con un nivel de {\it confianza} y
    {\it confort} que debe existir en el equipo de desarrollo. Significa no tener
    miedo de desperdiciar una tarde o un día de programación y empezar de nuevo si
    no si todo está bien. Significa poder estar en contacto con los instintos de
    uno mismo en relación con lo que funciona y lo que no.

    Valentía también significa responder a la retroalimentación concreta, actuando
    con base en las corazonadas de sus compañeros de equipo cuando ellos piensan 
    que tienen una forma más simple y mejor de obtener su objetivo.
\end{enumerate}

\vspace{1cm}
\subsubsection{ El Proceso de desarrollo ágil} 
El modelado es una palabra clave en los métodos ágiles El modelado ágil aprovecha
la oportunidad de crear modelos que pueden ser lógicos, como los dibujos de los
sistemas, o maquetas de tamaño natural como los prototipos. Un proceso ordinario
de modelado ágil podría ser el siguientes:\\

\begin{enumerate}
  \item Escuchar las historias de los usuarios por medio del cliente.
  \item Dibujar un modelo de flujo de trabajo lógico para apreciar las
    decisiones de negocios representadas en la historia de un usuario.
  \item Crear historias de usuarios con base en el modelado lógico.
  \item Desarrollar algunos prototipos de visualización. Para ello hay que mostrar
    los clientes el tipo de interfaz que tendrán.
  \item Usar la retroalimentación de los prototipos y los diagramas del flujo de 
    de trabajo lógico para desarrollar el sistema hasta crear un modelo físico de 
    datos.
\end{enumerate}

\vspace{1cm}
\subsection{Comparativa entre SDLC y la metodología ágil}\\ 

La metodología ágil es un enfoque moderno para el desarrollo de \textit{software} 
\footnote{Tesis Doctoral, \textit{Metodología Ágil de Desarrollo de Software Enfocado a Trabajos de Grado en Ingeniería} - autor: Gustavo Armando Rivera Sanchez.}, mientras que el SDLC \footnote{ \textit{SDLC}:Ciclo de Vida del Desarrollo de Software} es un método tradicional. ambas se utilizan para desarrollo y mantener software de manera eficiente.\\

La principal diferencia entre estas dos Metodologías es el enfoque. La metodología ágil se centra en la adaptación, la iteraciones y la mejora continua, mientras que el SDLC se basa en una planificación detallada, una estructura formal y un proceso escrito. Otro gran diferenciador es que la metodología ágil se basa en equipo auto-organizado, mientras que el SDLC se basa ne roles y responsabilidades claramente definidos.\\

Otras diferencias entre SDLC y la metodología ágil incluyen el tiempo de desarrollo, el control de calidad, el grado de documentación y la estabilidad del producto. La metodología ágil suele tener un tiempo de desarrollo más corto, un control de calidad más alto y una documentación más ligera. Por otro lado, el SDLC suele tener un mayor estabilidad del producto, ya que el proceso es más estructurado y formal.\\

Los investigadores ({\it Davis} y {\it Naumann}, 1999) desarrollaron un lista de (7)
estrategias que pueden mejorar las eficiencia: 

\begin{enumerate}
  \item Reducir el tiempo de los errores de la interfaz.
  \item Reducir el tiempo de aprendizaje del proceso y las pérdidas duales
    de procesamiento.
  \item Reducir el tiempo y esfuerzo requeridos para estructurar las tareas
    y aplicar formatos a las salidas.
  \item Reducir la expansión improductiva del trabajo.
  \item Reducir el tiempo y costo del almacenamiento, la investigación de datos
    y del conocimiento.
  \item Reducir el tiempo y costos de la comunicación y la coordinación.
  \item Reducir las perdidas debido a la sobrecarga humana de información.
\end{enumerate}


\subsubsection{Implementación mediante Metodologías ágiles}
\begin{enumerate}
  \item Adoptar la programación en pareja.
  \item Creación de prototipos y  desarrollo rápido.
  \item Fomentar las entregas pequeñas.
  \item Limitar el alcance en cada entrega.
  \item Permitir un cliente en el sitio.
  \item Usar cajas de tiempo (timeboxing).
  \item Apegarse a una semana de trabajo de 40 horas.
\end{enumerate}

\newpage
\section{ Conclusiones }
\vspace{1cm}
\begin{center}
  \begin{minipage}{14cm}
    Hemos visto que tanto los métodos ágiles como la interacción con los programadores, clientes, analistas y humano-computadora son fundamentales para el éxito de un proyecto. Los analistas de sistemas pueden utilizar prototipos para recopilar reacciones de los usuarios, sugerencias, innovaciones y planes de revisión para realizar mejoras al prototipo y modificar los planes del sistema con un mínimo de costo y de interrupciones. Al desarrollar un prototipo, se deben tener en cuenta cuatro alineamientos principales: trabajar en módulos administrables, crear el prototipo con rapidez, modificar el prototipo y hacer énfasis en la interfaz de usuario. Estas prácticas pueden ayudar a garantizar un proceso de desarrollo eficaz y eficiente que resulte en un sistema de calidad.
  \end{minipage}
\end{center}

\newpage
\section{ Bibliografías }

\begin{enumerate}
  \item \textbf{ \textit{ Análisis y Diseño de Sistemas } }\\
  (\textit {KENDALL \& KENDALL })\\
  \textit{ octava edición }

  \item \textbf{ \textit{Cognodata}} Principios de Metodología Ágil \\ 
  \href{https://www.cognodata.com/principios-metodologia-agile-desarrollo-proyectos}{www.cognodata.com}

  \item \textbf{\textit{Tesis Doctoral}}, \textit{Metodología Ágil de Desarrollo de Software Enfocado a Trabajos de Grado en Ingeniería} - autor: \textbf{Gustavo Armando Rivera Sanchez}.
  \href{https://repository.unilibre.edu.co/bitstream/handle/10901/23827/Metodolog%C3%ADa%C3%81gilDeDesarrolloDeSoftwareEnfocadoATrabajosDeGradoEnIngenier%C3%ADa.pdf?sequence=2&isAllowed=y}{Tesis Doctoral 2021}

\end{enumerate}

\vspace{1cm}

\end{document}
\end{document}

