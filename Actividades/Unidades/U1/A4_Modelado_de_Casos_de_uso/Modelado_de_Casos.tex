\documentclass[12pt]{article}
\usepackage[total={18cm,22cm},top=3cm,left=2cm,right=2cm]{geometry}
\parindent=0mm
\usepackage[utf8]{inputenc}

\usepackage{multicol}

\usepackage{hyperref}

\usepackage{graphicx}

\usepackage{amsmath}

\usepackage{tikz}
\usepackage{enumitem}
\definecolor{azulF}{rgb}{.0,.0,.3}
\newcommand{\cnumero}[2]{\tikz[baseline=(myanchor.base)]
\node[minimum size=0.2cm,circle,
inner sep=1pt,draw, #2, thick, fill=#2](myanchor)
{\color{white}\bfseries\fontsize{8}{8}#1};}

\newcommand*{\itembolasazules}[1]{\protect\cnumero{#1}{azulF}}

\usepackage{fancyhdr}
\pagestyle{fancy}
\fancyhf{}
%TODO
\fancyhead[L]{TEORÍA DE SISTEMAS}
%TODO
\fancyhead[R]{UNAH-VS}
\fancyfoot[C]{\thepage}

\renewcommand*\contentsname{Contenido}

\begin{document}
%*******************************************************
% Portada
%*******************************************************
\begin{titlepage}

  \begin{center}
    {\includegraphics[width=0.65\linewidth]{/home/xtriminio/.config/nvim/template/Imgs/unahWallpaper.png}\par}

    
  
    {\bfseries\Huge Universidad Nacional Autónoma de\\
                     Honduras en el valle de Sula \par}
  
    \vspace{1cm}
  
%TODO
    {\scshape\huge TEORÍA DE SISTEMAS\par}
  
    \vspace{1cm}
  
%TODO
    {\scshape\Large CONCEPTOS DE MODELADO DE CASOS DE USO}
  
    \vfill 
    {\large Catedrático:\par} %TODO
    {\large Ing. Tania Melissa Pineda Godoy \par} %TODO
  
    \vfill
    {\large Alumno: \par}
    {\large Jimmy Xavier Triminio Hernández - 20122008197 \par}

    \vfill
    {\large Febrero 2023\par} %TODO

  \end{center}

\end{titlepage}

%*******************************************************
% Contenido 
%*******************************************************
\newpage
\tableofcontents

%*******************************************************
% Introducción
%*******************************************************
\vspace{1cm}
\section{ Introducción }

Los conceptos principales del modelado de casos de uso
son una herramientas esenciales para desarrollar sistemas 
informáticos y procesos empresariales. Estos conceptos 
incluyen casos de uso, actores, puntos de extensión, 
flujo de eventos, diagramas de casos de uso, relaciones
de casos de uso y excepciones. Los casos de uso describen
los pasos necesarios para alcanzar objetivos 
específicos en un sistema, mientras que los actores
son usuarios o entidades externas que interactúan con
el sistema a través de los casos de uso. Los puntos de 
extensión marcan los puntos donde la funcionalidades
del sistema pueden ampliarse, mientras que los flujos 
de eventos describen los pasos necesarios para 
completar un caso de uso. Los diagramas de casos de uso
proporcionan una vista gráfica de los casos de uso, los 
actores y las relaciones entre ellos, mientras que las
relaciones de casos de uso describen la forma en que 
los casos de uso están relacionados entre sí. Finalmente, las excepciones se refieren a situaciones
inesperadas que pueden surgir durante el proceso de un
caso de uso.

%*******************************************************
% Contenido 
%*******************************************************
\vspace{1cm}
\section{Conceptos Principales de Modelado de Casos de Uso}

Los conceptos principales de modelado de casos de uso
incluyen casos de uso, actores, puntos de extensión,
flujos de eventos, diagramas de casos de uso, 
relaciones de casos de uso y excepciones.

Los casos de uso son los pasos necesarios para lograr un
objetivo específico en un sistema. Estos casos pueden
incluir acciones como realizar una compra, iniciar una
sesión o enviar un mensaje. Los actores son los 
usuarios o entidades externas al sistema, como un cliente
o un proveedor, que interactúan con el sistema a través 
de los casos de uso. Los puntos de extensión marcan los 
puntos donde pueden ampliarse la funcionalidad del 
sistema. Los flujos de eventos describen los pasos 
necesarios para completar un caso de uso.

Los diagramas de casos de uso proporcionan un vista 
gráfica de los casos de uso, los actores y las
relaciones entre ellos. Las relaciones de casos de uso
describen la forma en que los casos de uso están
relacionados entre sí. Las excepciones se refieren a 
situaciones que pueden producirse durante el proceso
de uso, como un error o una condición de frontera.\\

\textbf{Caso de Uso}:
\begin{itemize}
  \item  Es un \textit{artefacto} que define una secuencia de acciones que dan lugar a un resultado de valor observable.
  \item Los casos de uso \textit{proporcionan} una estructura una estructura para expresar requisitos funcionales en el contexto de procesos empresariales y de sistema.
\end{itemize}

%*******************************************************
% Biografia 
%*******************************************************
\section{ Biografía }
\vspace{1cm}

\begin{enumerate}
  \item \textbf{Diagrama de Casos de uso}\\
    \href{https://www.ionos.es/digitalguide/paginas-web/desarrollo-web/diagrama-de-casos-de-uso/}{www.ionos.es}

  \item \textbf{Definición de casos de uso}\\
    \href{https://www.ibm.com/docs/es/elm/6.0.3?topic=requirements-defining-use-cases}{www.ibm.com}
\end{enumerate}


\end{document}

